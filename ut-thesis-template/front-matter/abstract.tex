\chapter*{Abstract}\label{ch:abstract}
The genetic code is redundant, with most amino acids coded by multiple codons. 
In many organisms, codon usage is biased towards particular codons. 
A variety of adaptive and non-adaptive explanations have been proposed to explain these patterns of codon usage bias. 
Using mechanistic models of protein translation and population genetics, I explore the relative importance of various evolutionary forces in shaping these patterns.
This work challenges one of the fundamental assumptions made in over 30 years of research: codons with higher tRNA abundances leads to lower error rates.
I show that observed patterns of codon usage are inconsistent with selection for translation accuracy.
I also show that almost all the variation in patterns of codon usage in \emph{S. cerevisiae} can be explained by a model taking into account the effects of mutational biases and selection for efficient ribosome usage.
In addition, by sampling suboptimal mRNA secondary structures at various temperatures, I show that melting of ribosomal binding sites in a special class of mRNAs known as RNA thermometers is a more general phenomenon.