\documentclass[12pt]{article}

\usepackage{xspace}
\usepackage{lineno}
\usepackage{setspace}
\usepackage{graphicx}
\usepackage{subfigure}
\usepackage{float}
\usepackage{color}
\usepackage{caption}
\usepackage[margin=1in]{geometry}
\usepackage{natbib}
\usepackage{amsmath}


\begin{document}
\doublespacing
\linenumbers

\newcommand{\kluyveri}{\textit{L. kluyveri}\xspace}
\newcommand{\dubl}{\textit{C. dubliniensis}\xspace}
\newcommand{\gossypii}{\textit{E. gossypii}\xspace}
\newcommand{\ROC}{ROC SEMPPR\xspace}
\newcommand{\GC}{GC content\xspace}

\noindent RH: LANDERER ET AL.--- Intragenomic variation in codon usage
% put in your own RH (running head)
% for POVs the RH is always POINT OF VIEW
\bigskip
\medskip
\begin{center}

% Insert your title:
\noindent{\Large \bf Differences in Codon Usage Bias between genomic regions in the yeast \textit{Lachancea kluyveri}.}
\bigskip

% We don't use a special title page; the author information is entered
% like any other text.

% FOOTNOTES: We don't allow them in the manuscript, except in
% tables. Don't include any footnotes in the text.


\noindent{C\textsc{EDRIC} ~{L\textsc{ANDERER}}$^{1,2,*}$,
R\textsc{USSELL} {Z\textsc{ARETZKI}}$^{3}$,
\textsc{AND}
M\textsc{ICHAEL} A.~{G\textsc{ILCHRIST}}$^{1,2}$}

\end{center}

\vfill

{\small
\noindent$^{1}$Department of Ecology \& Evolutionary Biology, University of Tennessee, Knoxville, TN 37996-1610\\
\noindent$^{2}$National Institute for Mathematical and Biological Synthesis, Knoxville, TN 37996-3410\\
\noindent$^{3}$Department of Business Analytics \& Statistics, Knoxville, TN ~ 37996-0532 \\
\noindent$^{*}$Corresponding author. E-mail:~cedric.landerer@gmail.com
}

\vfill
\centerline{Version dated: \today}
\vfill
\newpage

\begin{abstract}
Codon usage has been used as a measure for adaptation of genes to their genomic environment for decades. 
The introgression of genes from one genomic environment to another may cause well adapted genes to be suddenly less adapted due to their signature of a foreign genomic environment.
The reflection of a foreign genomic environment in transferred genes can result in a large fitness burden for the new host organism.
%However, it is possible to observe the opposite, as genes can be exapted; for example if mutation bias in the donating organism matches selection bias in the receiving organism.
Here we examine the yeast \textit{Lachancea kluyveri} which has experienced a large introgression, replacing the left arm of chromosome C ($\sim 10 \%$ of its genome).
The \kluyveri genome provides an opportunity to study the adaptation of introgressed genes to a novel genomic environment and estimate the fitness cost such a transfer imposes.
The codon usage of the endogenous \kluyveri genome and the exogenous genes were analyzed, using \ROC which allows for the effects of mutation bias and selection bias on codon usage to be separated.
We found substantial differences in codon usage between the endogenous and exogenous genes, and show that these differences can be largely attributed to a shift in mutation bias from A/T ending codons in the endogenous genes to C/G ending codons in the exogenous genes.
Recognizing the two different signatures of mutation and selection bias improved our ability to predict protein synthesis rate by $17 \%$ and allowed us to accurately assess codon preferences.
In addition we utilize the estimates of mutation bias and selection bias gaines using \ROC to determine a potential source lineage, estimate the time since introgression and asses the fitness burden the introgressed genes represent showing the advantage of mechanistic models have when analyzing codon data.
%In order to identify a potential source of the exogenous genes, we estimated mutation and selection bias across 38 yeast lineages. Our initial comparison identified two candidates, \textit{Candida dubliniensis}  and \textit{Eremothecium gossypii}, as potential source linages.
%We excluded \dubl using orthogonal information on synteny.
%We also estimated that the exogenous genes introgressed $6.22\times10^8$ generation ago into the \kluyveri genome.
\end{abstract}
\newpage

\section*{Introduction}
\begin{itemize}
	\item A genes codon usage reflects the genomic environment it has evolved in.
	\begin{itemize}
		\item Mutation, selection, and drift are fundamental forces shaping the genomic environment.
		\item The efficacy of selection on codon usage is often assumed to be proportional to gene expression, where highly expressed genes show a greater level of adaptation than low expression genes.
		\item Codon usage in low expression genes on the other hand, is dominated by mutation driven bias. 
		\item Together, mutation driven bias - or mutation bias - and selection driven bias scaled by gene expression - or selection bias - shape codon usage in a genome; allowing us to describe the genomic environment in which genes evolve with respect to these terms.
		\item Estimating the influence of mutation bias and selection bias on a gene improves our understanding of its evolution; giving us the ability to describe it's history and making inferences of it's future.
	\end{itemize}
	\item Most studies implicitly assume that codon usage in a genome is the product of a single genomic environment.
	\begin{itemize}
		\item This assumption however, can be violated by horizontal gene transfer, introgression, or hybridization.
		\item The transfer of genes to a different genomic environment may cause them to be less adapted to the novel environment, with potentially large fitness consequences if the two genomic environments differ greatly in their selection bias, making such transfers less likely.
		\item In contrast, similarities in the genomic environment could, under certain circumstances even promote the transfer of genomic material. 
		\item Furthermore, if unaccounted for, introgressed genes may distort parameter estimates describing codon usage and cause us to conclude wrong codon
	\end{itemize}
	\item In this study we analyze the synonymous codon usage in the genome of \kluyveri, the earliest diverging lineage of the Lachancea clade.
	\begin{itemize}
		\item The Lachancea clade diverged from the Saccharomyces lineage prior to the whole genome duplication about 100 Mya ago.
		\item It appears that \kluyveri has experienced a large introgression of exogenous genes replacing the whole left arm of chromosome C. 
		\item This chromosome arm has a \GC  $\sim 13 \%$ higher than the endogenous \kluyveri genome.
	\end{itemize}
	\item Using \ROC allows us to describe the genomic environment genes have evolved in by separating effects of mutation bias and selection bias.
	\begin{itemize}
		\item We utilize the ability to distinguish between effects of mutation bias and selection bias to describe two genomic environments reflected in the \kluyveri genome, an endogenous and an exogenous environment.
		\begin{itemize}
			\item We find that the $13 \%$ difference in \GC can be attributed to  a shift in codon usage from A/T ending codon in the endogenous genes to C/G ending codons in the exogenous genes due to differences in mutation bias.
		\end{itemize}
		\item Recognizing the difference in codon usage between endogenous and exogenous genes allows us to infer the codon preference ($\Delta \eta$) for \kluyveri without the exogenous genes distorting our estimates.
		\begin{itemize}
			\item We also observe a relative improvement of $17 \%$ in our ability to predict protein synthesis rate. 
		\end{itemize}
	\end{itemize}
	\item In addition to improvements to model fitting, we show the utility of the separation of mutation bias and selection bias by:
	\begin{itemize}
		\item Determining a potential source lineage of the introgressed exogenous genes.
		\begin{itemize}
			\item Comparing the estimates of $\Delta M$ and $\Delta \eta$ for the exogenous gene region to 38 yeasts and identified ancestors of \gossypii and \dubl as most likely sources of the introgression.
			\item Using orthogonal information on synteny with the exogenous genes, left \gossypii as only potential source.
		\end{itemize}
		\item Estimating the time since introgression and the persistence of the signal using estimates of mutation bias.
		\item Estimating the cost of the introgression using estimates of selection bias.
	\end{itemize}
\end{itemize}

\section*{Results}
	
\begin{itemize}
	\item We compared model fits of \ROC to the full \kluyveri genome, and the separated endogenous and exogenous genes.
	\begin{itemize}
		\item We find that the partitioning of the \kluyveri genome into endogenous and exogenous genes is clearly favored ($\Delta AIC \sim 90,000$).
		\item We find that the disagreement in selection bias causes us to predict wrong codon preferences for seven amino acids if endogenous and exogenous genes are not treated separately.
	\end{itemize}
	\item The comparison of parameter estimates reveals large differences in mutation bias and smaller differences in selection bias between endogenous and exogenous genes.
	\begin{itemize}
		\item We find little agreement in mutation bias between endogenous and exogenous genes with only two amino acids (A,F) showing complete concordance.
		\item Mutation is biased towards A/T ending codons ($11/19$) in the endogenous genes and strongly biased towards C/G ending codons ($17/19$) in the exogenous genes.
		\item However, we find the same codon preference in endogenous and exogenous genes for nine amino acids. 
		\item We also observe a stronger bias in selection towards C/G ending codons in the exogenous genes. 
		\item Furthermore, recognizing the differences in mutation driven bias and selection driven bias between endogenous and exogenous genes improves our relative ability to predict protein synthesis rate by $17 \%$ ($\rho = 0.59$ vs. $\rho = 0.69$).
	\end{itemize}
	\item Inference of mutation bias $\Delta M$ and selection bias $\Delta \eta$ of 38 other yeasts revealed 33 species with similarities in selection bias and five species with similarities in mutation bias.
	\begin{itemize}
		\item Only four species that showed similar mutation bias (\gossypii, \dubl, and \textit{Sphaerulina musiva}, \textit{Yarrowia lipolytica}) showed a positive selection bias relationship.
		\item Only \gossypii and \dubl had a strong positive relationship in both mutation bias and selection bias. 
		\item We validated our candidate sources using orthogonal information on synteny.
		\item We find that synteny with the exogenous genes is limited to the Saccharomycetacease group, eliminating \dubl as potential source, leaving us with \gossypii.
		\item Using our estimates of mutation bias we estimated the age of the introgression to be about $6.22\times10^8$ generations.
		\item We predict that decay of the signature of the sources genomic environment to one percent of the \kluyveri genomic environment will take about $5.66\times10^9$ generations.
	\end{itemize}
	\item We estimate the fitness burden of the introgressed region on \kluyveri and compare it to the fitness burden at the time of introgression, assuming no change in the \gossypii genomic environment and constant amino acid usage.  	
	\begin{itemize}
		\item We find that the exogenous genes were a large fitness burden at the time of introgression and still represent a large reduction in fitness relative to the replaced endogenous genes.
	\end{itemize} 
\end{itemize}

\section*{Discussion}

\begin{itemize}
	\item We partitioned \kluyveri into endogenous and exogenous genes using information about a previously identified introgression event.
	\item After we inferred parameters describing codon usage using \ROC, we find that the two gene sets show difference in mutation bias and selection bias
	\begin{itemize}
	\item Endogenous genes tend to be generally  biased towards A/T ending codons while exogenous genes are biased towards C/G ending codons.
	\item We observe higher correlation between $\Delta \eta$ than $\Delta M$, nevertheless we find the optimal codon differs between endogenous and exogenous genes in nine out of 19 synonymous codon families. 
	\item Without recognizing the difference in codon preference we would have inferred the preferred codon in the \kluyveri genome wrong for seven amino acids.
	\end{itemize}
	\item We also improve our relative ability to predict protein synthesis rate when separating endogenous and exogenous genes by $17 \%$. 
	\item The comparison of $\Delta M$ and $\Delta \eta$ estimates from the exogenous genes to 38 other yeast lineages provided us with 33 yeast lineages showing a positive relationship in selection bias and five lineages showing a positive relationship in mutation bias.
	\begin{itemize}
		\item We expect differences in mutation bias to decay more slowly than differences in selection bias due to these differences being mostly neutral.
		\begin{itemize}
			\item Therefore yeasts with similar estimates of $\Delta \eta$ are all likely sources.
			\item However, the longer persistence of mutation bias should allow us to identify a source more reliably without the worry about signatures of selection to dissipate. 
		\end{itemize}
		\item Synteny of the exogenous region was consistent with eight Saccharomycetaceae lineages.
		\begin{itemize}
			\item Most of these eight species showed similarity in selection bias but not in mutation bias.	
			\item Only \gossypii showed both synteny with the exogenous region and a similar mutation and selection bias.
		\end{itemize}
	\end{itemize}
	\item We estimated the age of the introgression to be on the order of  $6.22\times10^8$ generation using our estimates of $\Delta M$ from the exogenous genes and \gossypii.
	\begin{itemize}
		\item The slower decay of mutation bias relative to selection bias also allowed us to estimate the time until the introgression will have decayed to be about $5.66\times10^9$ generations.
		\item Differences in selection bias are expected to have decayed earlier.
		\item This is consistent with the observation that HGT is more common between lineages with simlar codon preference, as most methods (CAI, tAI) are insensitive to differences in mutation bias, focusing only on selection.
		\item We acknowledge that our assumption about a constant genomic environment in the \gossypii genome is unlikely.
	\end{itemize}
	\item Assuming that the amino acid sequence has not changed over time, we infer the fitness burden at the time of introgression of the exogenous genes on \kluyveri compared to the original endogenous genes from our estimates of selection bias.
	\begin{itemize}
		\item Estimating the impact the exogenous genes had on the fitness of \kluyveri at the time of the introgression revealed that this event was very unlikely to reach fixation.
		\item However, we are not aware of any estimates of the frequency at which such large scale introgressions of genes with very different signatures of mutation and selection bias occur, which may be indicative that we only observe these events very rarely.
		\item We also show that the exogenous genes still represent a large decrease a fitness relative to the hypothesized ancestral endogenous genes.
	\end{itemize}	
	\item In conclusion, we show the usefulness of the separation of mutation bias and selection bias in this study of codon usage and we illustrate how a mechanistic approach like \ROC can be used for more sophisticated hypothesis testing in the future.
	\begin{itemize}
		\item We highlight potential pitfalls when estimating codon preference, as estimates can be biased by the signature of a second, historical genomic environment.
		\item In addition, we show how estimates of selection relative to drift can be obtained from codon data and used to infer the fitness cost of introgressed genes.
	\end{itemize}
\end{itemize}



\end{document}