\chapter*{Abstract}\label{ch:abstract}

Mathematical and statistical models are useful for describing and understanding observations in genetics and genomics.
These models have to constantly be updated to reflect current biological understanding.
As opposed to descriptive and phenomenological models, mechanistic models allow for the extraction of more biologically relevant information based on underlying principles.
Mutation, selection, and genetic drift are the three forces guiding evolution.
Mechanistic models rooted in population genetics principles allow us to determine how these forces shape observed data.
I demonstrate the usage of mechanistic models to relate protein coding sequences to their fitness landscapes and the evolutionary forces shaping them.
Using the yeast \textit{L. kluyveri}, I show the increased cost of protein synthesis due to a large scale introgression with mismatched codon usage.
Furthermore, I analyze site-specific selection on amino acids in the beta-lactamase protein TEM, which confers antibiotic resistance in \textit{E. coli} and related species. 


