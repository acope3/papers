\documentclass[12pt,draft]{article}


\usepackage{lineno}
\usepackage{setspace}



\begin{document}
\doublespacing
\linenumbers


\section*{Introduction}

\begin{itemize}
\item Huge influx of genome scale data sets across the tree of life
\item Why study CUB?
	\begin{itemize}
	\item CUB is shaped by mutation and selection
	\item informs about selection on translation efficiency.
	\item translation efficiency can mean ribosome pausing, error free translation (nonsense or missense errors).
	\item CUB potentially relates to co-translational folding. Tools allows to get at questions
	\end{itemize}
\item compare to CAI and tAI
	\begin{itemize}
	\item CAI and such require reference set of house-keeping gens for comparison of adaptation
	\item tAI uses tRNA copy number implying no other factors are involved
	\item We have population genetics component getting at $N_e * s$ with our model
	\end{itemize}
\item inference of expression as evolutionary mean unaffected by growth conditions and experimental noise
\item Bayesian MCMC method allows for priors and multilevel and hierarchy 
\end{itemize}



\section*{Results}
\begin{itemize}
\item Performance (simulated data)
	\begin{itemize}
	\item runtime by number of genes and number of mixtures
	\item confidence in parameters by number of genes
	\item estimation of phi values (gene ordering)
	\end{itemize}
\item s_epsilon as a measure of noise due to the lab conditions by eliminating technical 			noise. Use replicates to estimates technical noise

\item 
\end{itemize}


\section*{Discussion}

\end{document}