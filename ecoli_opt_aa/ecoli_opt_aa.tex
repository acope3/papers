\documentclass[12pt]{article}

\usepackage{xspace}
\usepackage{lineno}
\usepackage{setspace}
\usepackage{graphicx}
\usepackage{subfigure}
\usepackage{float}
\usepackage{color}
\usepackage{caption}
\usepackage[margin=1in]{geometry}
\usepackage{epstopdf}
\usepackage{natbib}

\usepackage{glossaries}

\makeglossaries
\newglossaryentry{geneticLoad}
{
    name=genetic Load,
    description={$sN_e$ relative to an optimal sequence}
}
\newglossaryentry{consensusSeq}
{
    name=consensus sequence,
    description={Sequence constructed based in the most frequent amino acid at each site}
}
\newglossaryentry{obsSeq}
{
    name=observed sequence,
    description={The input sequence alignment}
}



\begin{document}
\doublespacing
\linenumbers



\noindent RH: LANDERER ET AL.--- Estimating genetic load
% put in your own RH (running head)
% for POVs the RH is always POINT OF VIEW
\bigskip
\medskip
\begin{center}

% Insert your title:
\noindent{\Large \bf Estimating the genetic load of natural protein coding sequences using a phylogenetic framework.}
\bigskip

% We don't use a special title page; the author information is entered
% like any other text.

% FOOTNOTES: We don't allow them in the manuscript, except in
% tables. Don't include any footnotes in the text.
\begin{abstract}
\end{abstract}	



\noindent{C\textsc{EDRIC} ~{L\textsc{ANDERER}}$^{1,2,*}$,
B\textsc{RIAN} C.~ {O\textsc{MEARA}}$^{1,2}$,
\textsc{AND}
M\textsc{ICHAEL} A.~{G\textsc{ILCHRIST}}$^{1,2}$}

\end{center}

\vfill

{\small
\noindent$^{1}$Department of Ecology \& Evolutionary Biology, University of Tennessee, Knoxville, TN 37996-1610\\
\noindent$^{2}$National Institute for Mathematical and Biological Synthesis, Knoxville, TN 37996-3410\\
\noindent$^{*}$Corresponding author. E-mail:~cedric.landerer@gmail.com
}

\vfill
\centerline{Version dated: \today}
\vfill
\newpage


\printglossaries


\section*{Introduction}
\begin{itemize}
	\item Natural selection favors proteins with greater functionality.
	\begin{itemize}
		\item However, selection can be overpowered by mutation or drift causing proteins to move away from the optimum.
	\end{itemize}
	\item Genetic load is usually assessed relative to a predefined wild-type.
	\begin{itemize}
		\item One can, however, assess genetic load relative to the genotype encoding a function of interest most optimally.
		\item However, this requires to assess the fitness of each genotype. 
	\end{itemize}
	\item Previously, deep mutation scanning (DMS) experiments have been utilized to assess site specific amino acid fitness for a variety of proteins.
	\begin{itemize}
		\item However, these experiments are limited to fast growing organisms that can be manipulated under laboratory conditions, and proteins where a specific selection pressure can be applied.
		\item Furthermore, DMS experiments utilize prepared libraries containing each genotype (ignoring epistatis), causing extremely low effective population sizes. 
		\item Thus, while mutation does not play a role, genetic drift reduces the efficacy of selection dramatically making it necessary to apply extremely high selection pressures. 
	\end{itemize}
	\item We utilize SelAC, a phylogenetic framework, to assess the genetic load of naturally occurring sequence variation on the species level.
	\begin{itemize}
		\item SelAC is a mechanistic phylogenetic model rooted in population genetics, and estimates site specific selection from sequence data.
		\item SelAC does not assume a uniform stationary amino acid distribution across sites, thus allowing it to estimate the optimal amino acid for each position given the available sequence data.
		\item Furthermore, SelAC is applicable to the whole tree of life and not limited to fast growing organisms that can be manipulated under laboratory conditions.
	\end{itemize}
	\item We predict the site specific optimal amino acid from sequence alignments of TEM, a $\beta$-lactamase in E. coli and cytochrome b (CytB), a mitochondrial transmembrane protein in whales.
	\item We then assess the genetic load of naturally occurring sequences TEM and CytB relative to the predicted functionally optimal amino acid sequence.
	\begin{itemize}
		\item We compare our genetic load estimates for TEM to empirical DMS estimates and find an increase of genetic load.
	\end{itemize}
	\item Furthermore, we will illustrate how the strength of selection varies along the analyzed proteins.
\end{itemize}

\section*{Results}
\begin{itemize}
	\item We predicted the functionally optimal amino acid at each site from the observed sequence variation using SelAC.
	\begin{itemize}
		\item The observed TEM alignment shows a high percentage of homogeneous sites.
		\begin{itemize} 
			\item $68 \%$ of sites had only one codon present.
			\item $75 \%$ of sites encoded the same amino acid.
		\end{itemize}
		\item The observed CytB alignment shows a more codon heterogeneity but a similar homogeneity in amino acids.
		\begin{itemize} 
			\item $22 \%$ of sites had only one codon present.
			\item $78 \%$ of sites encoded the same amino acid.
		\end{itemize}
		\item We find that the predicted optimal amino acid sequence has high agreement with the observed consensus sequence of the alignment (TEM: $99 \%$, CytB: $95 \%$).
		\item In contrast, the experimentally obtained sequence estimate only has an agreement of $49 \%$ with the observed TEM consensus sequence.
		%\item Simulations based on the inferred optimal sequences showed that the we would not expect to the observed sequences to have evolved.
	\end{itemize}
	\item We assessed the genetic load of the observed sequences.
	\begin{itemize}
		\item We find that the genetic load of TEM differs greatly depending on the optimal amino acid sequence assumed.
		\begin{itemize}
			\item The genetic load of the observed sequences increases $3-20$ fold when using the experimentally inferred optimal sequence compared to the SelAC inferred optimal sequence.
			\item Besides the great variation that arises from the usage of different optimal amino acid sequences, we also find variation within each optimal amino acid sequence.
			\item E.g. $sN_e$ varies between $\sim0$ to $\sim-10$ for the SelAC optimal sequence and between $\sim -20$ to $\sim -27$ for the optimal sequence obtained from the DMS experiment.
		\end{itemize}
		\item We lack the ability to compare our estimates of genetic load for CytB as DMS experiment can not be performed on whales.
		\item We find a higher genetic load and greater variation in $sN_e$ for the CytB (not taken into account: sequences differ in length).
		\begin{itemize}
			\item $sN_e$ varies between $\sim -10$ to $\sim -35$.
		\end{itemize}
	\end{itemize}
	\item We are able to map variation in selection along the sequence and determine sites with higher contribution to genetic load.
	\begin{itemize}
		\item Increases in genetic load appear to be locally confined to a few regions among the TEM alignment but do not appear to be associated with any particular structural features.
		\item In contrast, CytB shows variation of genetic load across its whole sequence with a particularly strong increase in genetic load within the 5th transmembrane helix.
	\end{itemize}
	\item Previous work highlighted the advantages of DMS experiments for phylogenetic inferences.
	\item However, our estimates of genetic load of observed TEM sequences show that natural sequences would actually represent a large genetic load.
	\begin{itemize}
		\item The SelAC estimated optimal amino acid sequence outperformed the consensus sequence and the experimentally sequence explaining the data. 
		\item A second model selection was performed using phydms as an independent comparison.
		\begin{itemize}
			\item Model selection revealed that the main advantage of the DMS experiment comes from the fact that the input alignment is not needed to estimate amino acid preferences.
			\item While the experimentally inferred optimal sequence does a worse job explaining the observed sequences, model selection reveals that the improvement in likelihood does not justify the increased number of parameters required to run phydms with the SelAC or the consensus amino acid preferences.
		\end{itemize}
	\end{itemize}
\end{itemize}

\section*{Discussion}
\begin{itemize}
	\item We demonstrate the inference of site specific selection from protein coding sequence data using phylogenetics.
	\item We estimate the genetic load of natural occurring proteins relative to an inferred optimal amino acid sequence.
	\item The optimal amino acid at each site was inferred from the observed proteins and their phylogenetic relationship.
	\item In both cases, TEM and CytB, we find high agreement between the consensus sequence inferred by ignoring the phylogenetic relationship and the optimal sequence inferred using SelAC (TEM: $99 \%$, CytB: $95 \%$).
	\begin{itemize}
		\item The strong agreement between consensus sequence and estimated optimal sequence for both proteins can be seen as an indication that the phylogenetic relationship does not play a large role in the examined cases.
		\item However, such an assumption should not be made a priory.
		\item The similarity between consensus and predicted optimal sequence could be because the proteins are under stabilizing selection like the model assumes, because rate of shifts in the optimal amino acid sequence is low, or because not enough time has passed for shifts to occur, despite diversifying selection.
		\item The used alignments contain a high amount of homogeneous sites (TEM: $ 75 \%$, CytB: $78 \%$), thus these sites do not allow for the inferred optimal amino acid to deviate from the observed consensus.
	\end{itemize}
	\item In contrast, the experimentally inferred optimal amino acid sequence for TEM only has $49 \%$ agreement with the observed consensus.
	\begin{itemize}
		\item Assuming that this inferred sequence is free of any bias introduced by the experimental conditions, we could only come to the conclusion that the observed TEM sequences show either strong mal-adaptation or did not have enough time to evolve towards the optimal sequence.
		\item However, \textit{E. coli} has a large effective population size, estimates are on the order of $10^8$ to $10^9$ (Ochman and Wilson 1987, Hartl et al 1994).
		\item The large $N_e$ would allow \textit{E. coli} to effectively "explore" the sequence space.
		\item On the other hand, each mutation in the library used for the DMS experiments starts of with only a few copies, potentially biasing the results due to strong genetic drift.
	\end{itemize}
	\item The genetic load of the observed sequences was inferred relative to the optimal amino acid sequence estimated by SelAC.
	\begin{itemize}
		\item Both, CytB and TEM show variation in the genetic load represented by each observed sequence, CytB represents a higher genetic load than TEM.
		\item Most TEM sequences show a small genetic load, likely due to the high selection pressure on TEM due to its usage in chemical warfare between microorganisms.
		\begin{itemize}
			\item If the experimental sequence is assumed to be most optimal, the observed TEM proteins represent a high genetic load to the organism. 
			\item This is would be in conflict with a large effective population size and therefore high efficacy of selection.
			\item However, while this would make fixation unlikely, it would not be impossible.
			\item In addition, the experimental sequence was inferred based on small population sizes for each genotype and artificial selection pressure. 
		\end{itemize}
	\end{itemize}
	\item Genetic load varies across the sequence.
	\begin{itemize}
		\item For both proteins, variation of $sN_e$ across the sequence is not associated with any particular structural features but mostly with variation in the alignment.
		\item However, TEM shows increased genetic load near the binding site, and the highest genetic load is found in the last beta sheet of the protein.
		\item The genetic load is generally higher for CytB than for TEM, and like for TEM genetic load appears to increase around the binding sites.
		\item However, for both proteins, increases in genetic load are not limited to the binding sites.
	\end{itemize}
	\item DMS experiments have been incorporated into phylogenetic studies to supplement information on selection on amino acids.
	\begin{itemize}
		\item In contrast, this study shows that information on selection can be extracted from alignments of protein coding sequences.
		\item To no surprise, model selection clearly favored the optimal sequence inferred by SelAC when using SelAC, however, when using this sequence in phydms we find that the inferences from SelAC still explained the data better, but the increase in parameters did not merit the increase in likelihood.
		\item This highlights the limitations of DMS sequences to explain natural evolution.
	\end{itemize}

\end{itemize}



\end{document}







