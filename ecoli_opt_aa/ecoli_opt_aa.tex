\documentclass[12pt]{article}

\usepackage{xspace}
\usepackage{lineno}
\usepackage{setspace}
\usepackage{graphicx}
\usepackage{subfigure}
\usepackage{float}
\usepackage{color}
\usepackage{caption}
\usepackage[margin=1in]{geometry}
\usepackage{epstopdf}
\usepackage{natbib}
\usepackage{amsmath}

\begin{document}
\doublespacing
\linenumbers


\newcommand{\Lik}{\ensuremath{\text{\emph{L}}}\xspace}
\newcommand{\selacDMS}{\emph{SelAC}+DMS\xspace}
\newcommand{\phydms}{\emph{phydms}\xspace}
\newcommand{\selac}{\emph{SelAC}\xspace}
\newcommand{\ecoli}{\textit{E. coli}\xspace}
\newcommand{\gy}{\emph{GY94}\xspace}

\noindent RH: LANDERER ET AL.--- Estimating genetic load
% put in your own RH (running head)
% for POVs the RH is always POINT OF VIEW
\bigskip
\medskip
\begin{center}

% Insert your title:
\noindent{\Large \bf Experimentally informed phylogenetic models are biased towards laboratory conditions and can be improved upon by mechanistic models of stabilizing selection.}
\bigskip


\noindent{C\textsc{EDRIC} ~{L\textsc{ANDERER}}$^{1,2,*}$,
B\textsc{RIAN} C.~ {O\textsc{MEARA}}$^{1,2}$,
\textsc{AND}
M\textsc{ICHAEL} A.~{G\textsc{ILCHRIST}}$^{1,2}$}

\end{center}

\vfill

{\small
\noindent$^{1}$Department of Ecology \& Evolutionary Biology, University of Tennessee, Knoxville, TN 37996-1610\\
\noindent$^{2}$National Institute for Mathematical and Biological Synthesis, Knoxville, TN 37996-3410\\
\noindent$^{*}$Corresponding author. E-mail:~cedric.landerer@gmail.com
}

\vfill
\centerline{Version dated: \today}
\vfill
\newpage


\section*{Introduction}
\begin{itemize}
\item Phylogenetics plays an ever increasingly important role in biology.
  \begin{itemize}
  \item Examples or just a list of citations.
  \end{itemize}
\item Most commonly used methods
  \begin{itemize}
  \item Strengths
    \begin{itemize}
    \item Fast
    \item Easy to use (software packages)
    \end{itemize}
  \item Weaknesses
    \begin{itemize}
    \item Many ignore key forces in evolution.
    \item Nucleotide models?
      \begin{itemize}
      \item Mutation only: UNREST, GTR, JC.
      \item Mutation rates can vary between nucleotide positions.
      \item Use the same matrix for all site
      \end{itemize}
    \item Amino Acid models
      \begin{itemize}
      \item `selection' strictly phenomenological: PAM, BLOSSOM?        
      \item Used same matrix for all sites
      \end{itemize}
    \item Codon models
      \begin{itemize}
      \item Most popular one that includes selection (GY94 and derivatives)
        \begin{itemize}
        \item Commonly misinterpreted
        \item Restricted selection scenario: freq dependence.
        \end{itemize}
      \end{itemize}
    \item Mutation, AA, and codon models all end up with same AA equilibrium frequency for all sites.
    \item Biologists have long recognized that equilibrium frequencies, and thus the substition matrix responsible, can vary substantially between sites.
    \end{itemize}
  \item \citet{HalpernAndBruno1998} provide general model
    \begin{itemize}
    \item Can have distinct substitution matrix for each site.
    \item As a result requires $19 \times n$ parameters.
    \item Large number of parameters makes implementation unfeasible
    \end{itemize}
  \item Potential solutions to parameterization issue
    \begin{itemize}
    \item Use additional information: experiments via DMS
      \begin{itemize}
      \item Discuss DMS
      \end{itemize}
    \item Use better models
      \begin{itemize}
      \item Lartillot and colleagues mitigate this issue using a site categorization approach. (Mention in discussion as potential next step to avoid reviewers asking you to do this.)
      \item \SelAC solves parameterization problem by using a simplistic model of ...
        \begin{itemize}
        \item Discuss \SelAC's strengths and weaknesses
        \item Slower than standard
        \end{itemize}        
      \end{itemize}
    \item Use better models and additional data
    \end{itemize}
  \item Work done here
  \item Main findings
  \end{itemize}
\end{itemize}

  Numerous attempts to incorporate selection into phylogenetic models have been made.
	\begin{itemize}
		\item Phylogenetic inference of sequence relationship has long been focused on substitution rates and fixation probabilities.
		\item However, the importance of site specific equilibrium frequencies has long been noted.
		\item Models of site specific equilibrium frequencies tend to be unfeasible as they are very parameter rich.
		\item Incorporating selection from experimental sources, therefore, seems like an attractive option.
		\begin{itemize}
			\item Site specific selection on amino acid allows to incorporate heterogeneity of selection along the protein sequence into phylogenetic models.
		\end{itemize}
	\end{itemize}
	\item Independent fitness estimates have potential to greatly reduce number of parameters estimated from phylogenetic data.
	\begin{itemize}
		\item DMS generates estimates of site specific selection on amino acids for large amount of mutations in a single experiment.
		\item This allows for the fitting of complex site specific models to smaller data sets.
		\begin{itemize}
			\item Site specific selection on amino acids improves model fits.
		\end{itemize}
		\item Empirical selection estimates are not always available, and their application for phylogenetic inference is questionable.
		\begin{itemize}
			\item DMS experiments are limited to proteins and organisms that can be manipulated under laboratory conditions, greatly limiting their application in phylogenetics.
			\item Estimates depend on factors like initial library of mutants, leading to heterogeneous competing populations.
			\item The applied selection between the wild and the laboratory is likely to differ.
			\item Hilton et al. (2017) showed that the variation between DMS experiments can have a significant effect on their utility.
		\end{itemize}
		\item To understand the reliability of selection on amino acids infered by DMS to inform phylogenetic studies, we utilize a DMS experiment by Stiffler et al. (2016).
		\begin{itemize}
			\item TEM is found in gram-negative bacteria like \ecoli.
			\item The applied selection pressure was limited to ampicillin and focused on the sequence variant TEM-1.
			\item TEM, however, can confer resistance to a wide range of antibiotics, causing it to be of wide interest.
		\end{itemize}
	\end{itemize}
	\item Mechanistic model of site specific stabilizing selection on amino acids rooted in population genetics are an improvement over DMS.
	\begin{itemize}
		\item We used \phydms and \selac to assess the realibility of DMS estimates of selection to inform phylogenetic models.
		\begin{itemize}
			\item model selection prefered \selac over DMS estimates of selection.
		\end{itemize}
		\item Assesment of model adequacy via simulations highlights the inadequacy of experimentally inferred selection.
		\begin{itemize}
			\item Sequences similarity of optimal DMS sequence to observed consensus sequence lower than expected.
			\item Genetic load of observed sequences higher than expected.
		\end{itemize}
		\item Models fits informed by experimentally inferred selection improve model fit over conventional codon and nucleotide models but can be improved upon by \selac.
		\begin{itemize}
			\item We also compare model fit of the two codon models of site specific stabilizing selection to models fits of 227 other codon and nucleotide models using IQTree.
		\end{itemize}
	\end{itemize}
%		\item Comparison between \selac and empirical estimates of selection show that they are comparable when site specific selection is captured adequately by the experiment.
%		\item Furthermore, we show that extrapolating experimentally inferred selection between homologous proteins (TEM and SHV) can be inadequate.
\end{itemize}

\section*{Results}
\subsection*{Site Specific Stabilizing Selection on Amino Acids Improves Model Fit}
\begin{itemize}
	\item We evaluate model fits of models of site specific selection and 227 other codon and nucleotide models to 49 observed TEM sequences.
	\begin{itemize}
		\item All models of site specific selection improve model fit.
		\item Number of parameters estimated from phylogenetic data differs between \selac, and \selacDMS and \phydms, resulting in slightly worse AICc for \selac.
		\item However, \selac outperforms \phydms (Table \ref{tab:AIC}).
	\end{itemize}
\end{itemize}

\begin{table}[h]
  \centering
  \caption{Model selection, shown are the three models of stabilizing site specific amino acid selection (\selac, \selacDMS, \phydms) and the best performing codon and nucleotide model \citep{GoldmanAndYang1994, zharkikh1994}. 
  Reported are the log-likelihood $\log(\Lik)$, the number of parameters estimated $n$, AIC and $\Delta$AIC values.
  See Table X for results from all models we tested.}  
  \begin{tabular}{lrrrrrr}
    \hline
    Model		& $\log(\Lik)$ & n & AIC & $\Delta$AIC\\ \hline 
    \selac		& -1498 & 374& 3744&  0\\    
    \selacDMS 		& -1768 & 111& 3758& 14\\
    \phydms 		& -2061 & 102& 4326& 582\\
    \emph{SYM}+R2 		& -2230 & 102& 4663& 919\\
    \gy+F1X4+R2 		& -2243 & 102& 4690& 946\\ \hline
  \end{tabular}
  \label{tab:AIC}
\end{table}

\begin{itemize}
	\item We observe differences in topology between the model fit of \phydms and \selac.
	\begin{itemize}
		\item \selac is too slow for a topology search, however, due to the improved model fit of \selacDMS over \phydms, it is likely that the \phydms infered topology is inadequate due to the biased laboratory conditions.
		\item \gy is outperformed by several nucleotide model e.g. \emph{SYM}+R2, potentially indicating that frequency dependent selection is inappropriate for TEM.
		\item \selac model fit shows $84 \%$ of all evolution happening at the tips, while this reduces to $77 \%$ in the \phydms model fit.
	\end{itemize}
\end{itemize}

\subsection*{Assessing Adequacy of Laboratory and \selac Inferences of Site Specific Selection}
\begin{itemize}
	\item Assessing model adequacy as sequence similarity sequence of selectively favored amino acids and observed consensus sequence.
	\begin{itemize}
		\item Experimentally inferred selection is inconsistent with observed sequences.
		\item Experimentally inferred sequence of selectively favored amino acids has only $52 \%$ sequence similarity with the observed consensus sequence.
		\item \selac inferred sequence of selectively favored amino acids has $99 \%$ sequence similarity with the observed consensus sequence.
		\begin{itemize}
			\item It is tempting to assume that the consensus sequence will allways fair best, however, this would implicitly assume indepence between sequences.
			\item However, the high sequence similarity of the consensus sequence and the sequence of selectively favored amino acids is likely due to the high average sequence similarity between the 49 observed sequences of $98 \%$.
			\item Furthermore, in addition to providing an estimate of the selectively favored amino acid, \selac also allows to estimate the genetic load of observed amino acids.
		\end{itemize}
	\end{itemize}
\end{itemize}

\subsection*{Comparing Laboratory and \selac Inferences of Genetic Load}
\begin{itemize}
	\item Laboratory estimates predict large genetic load
	\begin{itemize}
		\item Simulations under DMS and \selac inferred selection were used to establish a baseline expectation.
		\item Assuming the site specific selection estimated by DMS, the observed TEM sequences represent an average sequence specific genetic load of $17.12$ or, equivalently, an average site specific load of $0.065$.
		\item This load is significantly larger than the simulated sequences with an average sequence specific load of $6.68$ or, equivalently, an average site specific genetic load of $0.025$
		\item In contrast, assuming the site specific selection estimated by \selac, the observed TEM sequences represent an average sequence specific genetic load of $6.4\times10^{-5}$ or, equivalently, an average site specific load of $2.4\times10^{-7}$.
		\item Again, the simulated sequences show a decreased genetic load with an average sequence specific load of $1.3\times10^{-5}$ or, equivalently, an average site specific genetic load of $4.8\times10^{-8}$.
	\end{itemize}
\end{itemize}

\subsection*{Comparing Laboratory and \selac Inferences of Site Specific Selection}
\begin{itemize}
	\item Distribution of genetic load differs between DMS inferred site specific selection and \selac inferred site specific selection.
	\begin{itemize}
		\item Assuming the site specific selection estimated by DMS, 111 sites have a genetic load of 0.
		\item Assuming the site specific selection estimated by \selac, 207 sites have a genetic load of 0.
		\begin{itemize}
			\item In general, it is not surprising to find a large number of sites with 0 genetic load as many sites (X \%) show no variation in the observed amino acid.
		\end{itemize}
		\item The selection estimates from DMS and \selac agree for $107$ sites at which no genetic load is found.
		\item Thus, for $100$ sites \selac does estimate a genetic load of 0 but DMS does estimate non-zero genetic load, the inverse is true for four sites.
		\begin{itemize}
			\item A closer look at the $100$ sites for which \selac does estimate a genetic load of $0$ but DMS does estimate a non-zero load revealed that all $100$ sites display a significant difference in likelihood between the \selac and DMS estimated optimal amino aicd.
			\item These $100$ sites show a significantly ($p = 3\times10^{-13}$) higher mean genetic load under the DMS estimates than the remaining $163$ sites of $0.0157$ and $0.003$, respectively, indicating that DMS represents the evolution of TEM particularly badly at these sites.
		\end{itemize}
		\item For the $52$ sites where both, DMS and \selac, estimate a non-zero genetic load we a correlation of $\rho = 0.247$, explaining $6 \%$ of the variation in the empirical selection estimates, when compared on the log scale.
		\begin{itemize}
			\item In $26$ cases \selac and DMS estimate the same optimal amino acid.
			\item The remaining cases all show a significant difference in likelihood between the \selac and DMS infered optimal amino acids.
			\item The $26$ cases in which the infered optimal amino acid differs, we observe a significantly higher mean genetic load ($p = 2\times10^{-5}$) than in the remaining $26$ sites of $0.0158$ and $0.004$, respectively, for which \selac and DMS estimate the same optimal amino acid
		\end{itemize}
	 %\item Where are these sites?
	\end{itemize}
\end{itemize}

%\subsection*{Comparing \selac Inferences of Site Specific Selection for Homologous Sequences TEM and SHV}
%\begin{itemize}
%	\item Site specific genetic load for TEM and SHV is not correlated ($\rho = 0.006$) and , despite similar $\alpha_G$
%	\begin{itemize}
%	 \item Exclusing site with no genetic load and calculating the correlation on the log scale lead to a correlation coefficient of $\rho = 0.22$. 
%	\end{itemize}
%	\item Greatest difference is observed in the physicochemical properties, specifically $\alpha$.
%	\item No significant differences are observed in average genetic load between secondary structure elements (Table \ref{tab:selection}).
%\end{itemize}

%\begin{table}[h]
%  \centering
%  \caption{Efficacy of selection ($G$) and genetic load for TEM and SHV, and separated by secondary structure. $G$ was estimated as a truncated variable with an upper bound of 300.}
%  \begin{tabular}{llrrrrr}
%    \hline
%    & & & \multicolumn{2}{c}{$G$} & \multicolumn{2}{c}{Genetic Load $L_i$} \\ 
%    Protein & Secondary Structure & \# Residues	& \multicolumn{1}{c}{Mean} & \multicolumn{1}{c}{SE} & \multicolumn{1}{c}{Mean} & \multicolumn{1}{c}{SE} \\ \hline 
%    TEM	&		& 263 & 219.3 & 7.5  & $15.9\times10^{-8}$ & $6.5\times10^{-8}$ \\
%    &Helix 		& 113 & 206.1 & 12.4 & $17.5\times10^{-8}$ & $13.1\times10^{-8}$ \\
%    &$\beta$-Sheet 	&  48 & 238.6 & 15.8 & $ 6.8\times10^{-8}$ & $2.9\times10^{-8}$ \\
%    &Unstructured 	& 102 & 224.8 & 11.4 & $18.6\times10^{-8}$ & $8.1\times10^{-8}$ \\
%    &Active/Binding Sites 	&   5 & 202.6 & 62.2 & $0.01\times10^{-8}$& $0.01\times10^{-8}$ \\ \hline
    
%    SHV&		& 263 & 244.9 & 6.8  & $4.0\times10^{-8}$ & $1.9\times10^{-8}$ \\
%    &Helix		& 102 & 234.6 & 11.5 & $7.3\times10^{-8}$ & $4.8\times10^{-8}$ \\
%    &$\beta$-Sheet 	&  66 & 253.1 & 12.8 & $2.1\times10^{-8}$ & $1.1\times10^{-8}$ \\
%    &Unstructured	&  95 & 224.7 & 11.4 & $1.5\times10^{-8}$ & $0.6\times10^{-8}$  \\
%    &Active/Binding Sites	&   5 & 239.9 & 60.0 & $1.5\times10^{-8}$ & $1.5\times10^{-8}$ \\ \hline
%  \end{tabular}
%  \label{tab:selection}
%\end{table}


\section*{Discussion}
\begin{itemize}
	\item We evaluate how well experimental selection estimates obtained by DMS explain natural sequence evolution and compare it to a novel phylogenetic framework, \selac.
	\begin{itemize}
		\item Previous work has shown that DMS selection estimates can improve model fit over classical approaches like GY94 and our work confirms this.
		\item However, model selection shows that the \selac model fit and the corresponding fitness estimates are favored over DMS estimates (Table \ref{tab:AIC}).
	\end{itemize}

	\item Adequacy of the DMS selection has previously not been assessed.
	\begin{itemize}
		\item The amino acid sequence with the highest fitness estimated using DMS has only $49 \%$ sequence similarity with the observed consensus sequence.
		\item In contrast, the SelAC estimate has $99 \%$ sequence similarity (Figure \ref{fig:sim_seqs_cons}). 
		\item In addition, we find evidence that experimental estimates of selection do not represent evolution in the wild.
 		\begin{itemize}
			\item Due to artificial selection environment; Heterogeneous population, very large $s$. 
			\item Only one antibiotic used, maybe a mixture of antibiotics would better reflect natural evolution.
			\item Lack of repeatability between labs introduces further problems (Firnberg et al 2014 vs. Stifler et al. 2016).
		\end{itemize}
	\end{itemize}

	\item Assuming that the DMS selection inference adequately reflects natural evolution, the observed TEM sequences are either maladapted or were unable to reach a fitness peak.
	\begin{itemize}
		\item However, \textit{E. coli} has a large effective population size, estimates are on the order of $10^8$ to $10^9$ (Ochman and Wilson 1987, Hartl et al 1994).
		\item The large $N_e$ would allow \textit{E. coli} to effectively "explore" the sequence space, thus suggesting that the TEM sequences are mal-adapted according to the DMS estimates.
		\begin{itemize}
			\item With a mutation rate of $2.54\times 10^{-10} \times 789 = 2\times 10^{-7}$ mutations per generation for TEM (Lee et al. 2012), we expect between $\mu N_e = 10^1$ and $10^2$ new mutations per generation of which on average XXX \% are advantages per site.
			\item Our simulations of sequence evolution with various $N_e$ values and the DMS fitness values show that we would expect higher adaptation even with much smaller $N_e$ (Figure \ref{fig:dms_sim}).
		\end{itemize}
		\item In addition, with an average site specific selection $0.085$, we would expect that mutations fix on average between $(4/|s|)\times \ln(2N_e) \approx 1200$ and $1300$ generations assuming $N_e$ to be on the order of $10^8$ to $10^9$ (Crow and Kimura 1970).
		\item As \textit{E. coli} doubles every 15 hours in the wild (Gibson et al. 2018), we would therefore expect that a mutation with an average $s = 0.085$ sweeps through the population of size $10^9$ in $\sim 1.5$ years.
		\begin{itemize}
			\item This sweep would only accelerate with reduced $N_e$ due to e.g. isolation between populations.
		\end{itemize}
	\end{itemize}
	
	\item The evidence derived from population genetics theory has us expecting the observed sequences to be at the selection-mutation-drift equilibrium, which is not the case if we assume the DMS inference of selection.
	\begin{itemize}
		\item Estimates of selection obtained from \selac, in contrast, show the observed sequences to be have high fitness.
		\begin{itemize}
			\item The average site specific genetic load estimated by \selac is four orders of magnitude lower than the average site specific load esimated using DMS ($2.4\times10^{-7}$ vs. $0.065$).
		\end{itemize}
		\item We find the majority of sequences near the optimum, indicating that the \selac estimates are consistent with theoretical population genetics results.
		\item Taken together, it appears that DMS reflects the selection on the TEM sequence with respect to only one antibiotic, which seems appropriate to model selection in a hospital environments but not when the interest lies in the evolution of TEM in the wild.
	\end{itemize}
	
	\item In addition to the result that \selac better explains the evolution of observed sequences in the wild, \selac has the advantage that it can be applied to any protein coding sequence alignment, however, is not without flaws itself.
	\begin{itemize}
		\item Like DMS and most phylogenetic models, \selac assumes site independence.
		\item \selac is a model of stabilizing selection, in contrast to e.g. GY94 which is a model of frequency dependent selection.
		\begin{itemize}
			\item Since TEM plays a role in the chemical warfare with conspecifics and other microbes, some sites may be under frequency dependent selection.
		\end{itemize}
		\item In addition \selac assumes that selection follows the same distribution for all sites.
		\begin{itemize}
			\item However, the distribution of selection could differ for sites in the different secondary structure types.
			\item Similarly, active sites may not follow the assumed distribution.
		\end{itemize}
		\item \selac also assumes that selection is proportional to the distance of amino acids in physicochemical space. 
		\begin{itemize}
			\item In this study, we defaulted to the properties described by Grantham (1974) polarity, composition, and molecular volume, however, many other distances are available which may improve model fit.
		\end{itemize}
	\end{itemize}
	\item Low sequence variation in the TEM may be cause for concern as it could be misinterpreted by the model as stabilizing selection because of the short branches.
	\begin{itemize}
		\item However, population genetics indicate the newly introduced mutations would sweep rapitly through the population if they provide a strong fitness advantage.
	\end{itemize}
	\item In conclusion, DMS experiments have been proposed to supplement information on selection on amino acids in phylogenetic studies.
	\begin{itemize}
		\item This study shows that information on selection can be extracted from alignments of protein coding sequences using a carefully constructed model of stabilizing selection rooted in first principles.
		\item Further, we highlight the bias of laboratory inferences of selection and suggest to focus effors in improving phylogenetic inference on the development of more realisitc models.
	\end{itemize}
\end{itemize}



\end{document}







