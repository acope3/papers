\documentclass[12pt]{article}

\usepackage{setspace}
\usepackage[margin=1in]{geometry}
\begin{document}
\doublespacing



\noindent RH: LANDERER ET AL.--- Estimating genetic load
% put in your own RH (running head)
% for POVs the RH is always POINT OF VIEW
\bigskip
\medskip
\begin{center}

% Insert your title:
\noindent{\Large \bf Phylogenetic model of stabilizing selection is more informative about site specific selection than extrapolation from laboratory estimates.}
\bigskip

\noindent{C\textsc{EDRIC} ~{L\textsc{ANDERER}}$^{1,2,*}$,
B\textsc{RIAN} C.~ {O\textsc{MEARA}}$^{1,2}$,
\textsc{AND}
M\textsc{ICHAEL} A.~{G\textsc{ILCHRIST}}$^{1,2}$}

\end{center}

\vfill

{\small
\noindent$^{1}$Department of Ecology \& Evolutionary Biology, University of Tennessee, Knoxville, TN 37996-1610\\
\noindent$^{2}$National Institute for Mathematical and Biological Synthesis, Knoxville, TN 37996-3410\\
\noindent$^{*}$Corresponding author. E-mail:~cedric.landerer@gmail.com
}

\vfill
\centerline{Version dated: \today}
\vfill
\newpage




\section*{Introduction}
\begin{itemize}
 \item Phylogenetic inference of sequence relationship was long focused on rates of substitutions.
 \begin{itemize}
  \item Focus has shifted towards site specific equilibrium frequencies (HB98, Bloom2014, ...)in the last 20 years.
  \item Such models however, tend to be not feasible as they are to parameter rich.
  \item Inference of site specific selection on amino acids from laboratory experiments e.g. DMS is therefore appealing.
 \end{itemize}
 \item Incorporation of external information on site specific selection on amino acids allows for the fitting more complex models.
  \begin{itemize}
   \item This comes with a loss of generality as DMS experiments are limited to fast growing organisms that can be manipulated under laboratory conditions.
   \item Strong artificial selection and very heterogeneous population with a lot of competing genotypes are a potential source of bias.
   \item In the case of TEM, the application of only one very specific antibiotic is unlikely evolutionary history, may reflect modern hospital environments.
  \end{itemize}
 \item In this study we will assess how adequate DMS inference of site specific selection on amino acids is using TEM and provide an alternative, more generally applicable solution.
 \begin{itemize}
  \item Simulations under the DMS inferred site specific selection on amino acids show that we would not expect to observe the natural TEM variants; revealing the inadequacy of DMS.
  \item We show that models fits achieved by the incorporation of DMS experiments can be improved upon using a hierarchical phylogenetic framework of stabilizing selection, SelAC.
  \item We further show that extrapolation even between sequences (TEM and SHV) with related function can be inadequate.
 \end{itemize}
\end{itemize}

\section*{Results}

\begin{itemize}
 \item Model selection shows that DMS can improve phylogenetic inference.
 \begin{itemize}
   \item phyDMS improved model fit to 49 TEM sequences by 142 log(likelihood) units
   \item number of parameters comparable to GY94 and others despite complex description of fitness landscape thanks to experimental estimates.
 \end{itemize}
  
  
 \item Lab inferences of selection (DMS) are inconsistent with natural sequence evolution.
 \begin{itemize}
   \item The inferred fitness landscape does not reflect observed sequences.
   \begin{itemize}
    \item The optimal amino acid sequence inferred by DMS only shows $49 \%$ sequence similarity with the observed sequences.
   \end{itemize}
 \item Observed sequences unlikely under the lab inferred fitness landscape.
   \begin{itemize}
    \item We would expect about half of the observed fitness burden.
    \item Sequence similarity is expected to be about $\sim 70 \%$.
   \end{itemize}
 \item Estimates of selection coefficients do not represent natural evolution.
   \begin{itemize}
    \item Due to artificial selection environment; Heterogeneous population, very large $s$. 
    \item Only one antibiotic used, maybe a mixture of antibiotics would better reflect natural evolution.
   \end{itemize}
 \end{itemize}
 
 \item SelAC better explains observed sequences than DMS and other models.
 \begin{itemize}
  \item Model selection shows that SelAC outperforms phydms (only for AIC).
  \item Model adequacy shows that SelAC better represents the observed sequences.
 \end{itemize}
 
 \item SelAC is a more general approach, applicable to all protein coding sequences.
 \begin{itemize}
  \item Application of SelAC to TEM, site specific estimates of aa fitness.
  \begin{itemize}
   \item most sites show the estimated optimal amino acid.
   \item We find that selection against used amino acids is clustered and locally confined.
  \end{itemize}
  \item Comparison between TEM and SHV reveals that extrapolation is not always a good idea.
  \begin{itemize}
   \item Site specific G terms for TEM and SHV are only weakly correlated ($\rho = 0.17$), despite similar $\alpha_G$.
   \item Greatest difference is observed in the physicochemical properties, specifically $\alpha$.
  \end{itemize}
 \end{itemize}
\end{itemize}

\section*{Discussion}

\end{document}
